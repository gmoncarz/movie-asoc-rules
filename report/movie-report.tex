%\documentclass[12pt]{article}
\documentclass[journal]{IEEEtran}
\usepackage[utf8]{inputenc}
\usepackage{listings}
\usepackage{lipsum}
\usepackage{graphicx}
\usepackage[spanish]{babel}

\begin{document}

\title{Trabajo Práctico Final de Reglas de Asociacion}
\author{
	Keuroghlanian, Alejo \\
	\and
	Krauthamer, Diego \\
	\and
	Moncarz, Gabriel}
\maketitle % this produces the title block

\begin{abstract}
TO BE DONE
\end{abstract}

\begin{IEEEkeywords}
Minería de datos
Reglas de asociación
Peliculas
\end{IEEEkeywords}

\section{INTRODUCUCCIÓN}

TO BE DONE





\section{Datos}
En la presente sección se analiza losdatos originales, como fueron enriquesidos,
como se los ha manipulado y las transformaciones realizadas. También se explica como
todo el procesamiento se ha automatizado con intenciones de que este mismo análisis 
pueda realizarse en caso de tener tanto datasets distintos, como se desee incorporar
nuevas entradas a los datos existentes.

\subsection{Fuente de datos}
El punto de partida del presente trabajo son  archivos relacionados con la información
sobre pelícuas y rating de películas. Los archivos en cuestión son 3
\begin{itemize}
	\item Peliculas: contiene 3.883 películas a ser analizadas. La información sobre cada
		una de ellas es:
		\begin{itemize}
			\item Código de película
			\item Nombre de la película
			\item Lista de géneros a la que pertenece
				\begin{lstlisting}[frame=single]
1::Toy Story (1995)::Animation
            |Children's|Comedy
2::Jumanji (1995)::Adventure
           |Children's|Fantasy
3::Grumpier Old Men (1995)
              ::Comedy|Romance
4::Waiting to Exhale (1995)
              ::Comedy|Drama
5::Father of the Bride Part II 
              (1995)::Comedy
				\end{lstlisting}
		\end{itemize}


	\item Usuarios: contiene 6.040 registros de usuarios distintos. La información de cada
		fila es
		\begin{itemize}
			\item código de usuario
			\item sexo (masculino o femenino)
			\item rango de edad
			\item ocupación o profesion
			\item código postal del usuario\footnote{Todos los usuarios analizados residen 
			  en los Estados Unidos}
				\begin{lstlisting}[frame=single]
1::F::1::10::48067
2::M::56::16::70072
3::M::25::15::55117
4::M::45::7::02460
5::M::25::20::55455
				\end{lstlisting}
		\end{itemize}

	\item Rating: raring de distintos usuarios sobre diversas películas.\footnote{Cabe
		destacar que cada usuario da su calificación sobre las películas que él desea. El
		usuario no esta obligado a completar una cantidad mínima de calificaciones.}

		\begin{itemize}
			\item Usuario que realiza la evaluación
			\item Pelicula evaluada
			\item Rating/evaluación asignada
			\item tiempo: momento en que es evaluada

        \begin{lstlisting}[frame=single]
1::1193::5::978300760
1::661::3::978302109
2::3108::3::978299712
2::3035::4::978298625
5::3081::3::978243054
5::377::4::978245999
        \end{lstlisting}
    \end{itemize}

\end{itemize}

Estos 3 archivos son, en resumen, los datos crudos utilizados. 
Todo el tratamiento de datos posterior, enriquesimiento de información como los
análisis y conclusiones finales, estan basados en el contenido de estos 3 archivos
fuentes.

\subsection{Preprocesamiento}
TO BE DONE
\subsection{Discretización de datos}
TO BE DONE
\subsection{Obtención de datos externos}
TO BE DONE
TODO: hablar del cache de peliculas
\subsection{Proceso de automatización}
TO BE DONE
\subsection{Generación de datasets}
TO BE DONE
TODO: Hablar del formato de los datasets (Por transacciones). 
TODO: Explicar todos los datasets que se generaron
TODO: Explicar que los datasets se generan dinámicamente
\subsection{Resumen}
TO BE DONE




\section{Procesamiento de datos}
En esta sección se describe como se procesan los datos generados en la sección
previa, con expectativas de encontrar reglas de asociación que permita hacer el
analisis exploratorio objetivo.

En primera instancia, con intenciones de tener una idea general del problema,
se intenta cargar los set de datos más genericos generados. Se busca explorar
que reglas genera, que tipos de reglas, la cantidad de reglas generadas, como
también cuales son los valores de soporte y confianza que retorne una cantidad
de reglas aceptables y útiles. 

\subsection{Weka 3.7.11}
Se intenta hacer el analisis de reglas de asociaciones con Weka. Este sofware por defecto
no lee archivos de transacciones, donde una transacción puede ser representada en varias filas,
 sino que usa el formato donde todas las variables esta definidas en una fila. Para convertir
de un formato a otro, se utiliza el filtro 
Denormalize\footnote{http://weka.sourceforge.net/packageMetaData/denormalize/index.html}
\footnote{El filtro Denormalize requiere la versión de Weka 3.7.2. Este es el motivo por el que 
el procesamiento se usa mediante una versión en desarrollo de Weka y no una estable}. 

Una vez que se tienen los datasets listo para ser procesados en Weka, se intentan generar reglas
de asociación exploratorias con Apriori. Se comienza con niveles de soporte y confianza elevados,
aproximadamente de 0,9 en ambos, y como no generan reglas, se comienza a reducir ambos parámetros
en 0,10. A medida que el algorítmo comienza a generar reglas, los tiempos de ejecución aumentan
de forma exponencial. Esto ha impedido generar reglas de asociación no-triviales. Para obtener
5 reglas básicas, Weka demanda más de 1 hora de procesamiento. Esto complica realizar un análisis
exploratorio supervisado por el usuario, objeto de interes en esta desarrollo.

Intentando descubrir los motivos de la lentidud de Weka, se han observado 2 factores fundamentales:
\begin{itemize}
	\item Weka utiliza sólo 1 de los 4 nucleos disponibles, sin paralelizar el procesamiento
	\item El filtro Denormalize genera una matriz esparsa con Verdaderos y Falsos, pero \textbf{no} con
	datos faltantes para los falsos. 
\end{itemize}

El primer punto no tiene salvación simple. Agregar multiprocesamiento al módulo de apriori esta
fuera del alcanze del presente análisis.

El segundo punto tiene dos inconvenientes. Por un lado, como ya fue explicado, es el tiempo de 
procesamiento que demanda este formato. El segundo es que la mayor parte de las reglas de asociación
generadas, son relacionadas a la no existancia de items, asunto que tiene sentido común.

Para resolver este segundo inconveniente se aplica un segundo filtro donde aplica datos faltantes 
a los registros falsos. De esta forma se tiene una representación precisa del problema. Al intentar 
hacer esto, Weka en primer instancia se queda sin memoria. Para ello se modifican los parámetros de
la virtual machine de Java para asignarle mas memoria. Con 1 GB de memoria, el procesamiento de
Apriori devuelve reglas de asociación básicas en un tiempo razonable (cercano a los 30 segundos). 
Sin embargo, cuando a Weka se le piden más reglas de asociaciones y se le disminuyen los rangos de 
soporte y confianza, el algoritmo requiere una cantidad de tiempo superior a una hora. Estas unidades
de tiempo de procesamiento dificultan poder hacer un análisis introductorio al problema general,
motivo por el que se decide intentar con otro software.


\subsection{R 3.1.0}
A la problemática inicial se le suma obtener resultados en tiempo razonables. 

Se procede a realizar el mismo procesamiento de Weka usando R. Se usa el módulo arules. Éste 
módulo soporta que las entradas tengan un formato de transacciones\footnote{formato donde un 
campo define el número de transacción y otro campo define el id del producto}
como de basket\footnote{formato donde en una fila se listan todos los productos de la transacción}.
Durante todo el procesamiento y análisis se usa el formato por transacciones.

R muestra ser eficiente a la hora de generar reglas de asociaciones. R demora 130 segundos para
cargar inicialmente la matriz esparsa de datos. Una vez que el dataset esta cargado, todas las
corridas de Apriori se han ejecutado en menos de 1 segundo. El dataset de referencia es el
dataset mas grande utilizado durante el trabajo de investigación. Éste tiene
\begin{itemize}
	\item 8.870.000 líneas
	\item 888.000 transacciones
	\item peso: 220 MB
	\item contenido: información de películas, usuarios, actores y directores.
\end{itemize}

En primer termino genera desconfianza  que los tiempos de ejecución de Apriori en R 
sean tan rápidos,
más aún al contrastarlos con Weka. 
El motivo es porque arules maneja eficientemente la matriz esparsa de transacciones y
porque no almacena los items inexistentes como lo hace Weka.


\subsection{Elección de los parámetros de Apriori}

Tener salidas de apriores con tiempos tan cortos, permite que puedan analizarse 
varios dataset, con distintos contenidos, y poder variar los parámetros de apriori gradualmente
hasta conseguir los valores adecuados que generen reglas de asociación interesantes.

Esto permite hacer un análisis exploratorio inicial sin mayores dificultades. Más aún, todo el 
análisis del presente trabajo puede hacerse corriendo Apriori sobre todos los datasets generados
con un bajo soporte, una baja confianza, y luego ordenar la salida por soporte y confianza.
Una vez que se tiene las reglas de salida, sólamente resta seleccionar las mejores reglas 
no triviales con mejores soporte y confianza conjuntos.

Pese a que la librería arules tiene un exelente desempeño en sus tiempos de ejecución, se intenta
encontrar los parámetros de Apriori adecuados, para que cuando se tenga un set de datos tan pesado
que el algoritmo demore un tiempo considerable en ser procesado, 
solo sea necesario hacer unas pocas iteraciones y
no iterar masivamente hasta encontrar reglas de asociaciones interesantes. En pocas palabras,
se intenta hacer un análisis exploratorio inteligente y no de fuerza bruta.

Para esto se utiliza el paquete arulesviz, librería que ayuda a mostrar las 
distintas reglas en un gráfico de dispersión. El gráfico muestra las reglas
en los distintos niveles de soportes y confianzas. Entonces, un análisis inteligente
puede hacerse corriendo una vez Apriori con bajos niveles de soporte y confianza, y 
en base a la nube de puntos se puede elegir estos argumentos para que genere 
una  cantidad aproximada de reglas deseada. Luego, se deben observar las reglas mismas para ver
si los resultados son los esperados, y en caso de ser necesario, correr nuevamente
Apriori con valores mas ajustados u holgados.

En el grafico \ref{arulesviz} muestra la nube de puntos de uno de los dataset de películas.
Al comenzar a estudiar reglas de asociación se tiene la expectativa de poder generar
reglas con niveles de soporte cercanos al 90\%, pero tan solo conver el gráfico
se puede apreciar que hay muy pocas reglas con soporte mayor al 10\%, y que la mayoría
estan recién cercanas al 0,5\%

\begin{figure}[ht!]
\centering
\includegraphics[width=90mm]{arulesviz.png}
\caption{Gráfico de dispersion de reglas generadas}
\label{arulesviz}
\end{figure}

Al correr Apriori con un minsup del 10\%, para ver cuales son estas reglas tan
fuertes respecto de las demas, se observan que son reglas cuyo lado izquiero
esta vacío, o reglas tan triviales como \{sexo$=$hombre\} $=$$>$ \{rating$=$high\}

Es por esto que, para el análisis de peliculas realizado en este informe, se han
realizado, en primera instancia, 
 corridas de Apriori con soportes mínimos en el rango de \[0,009 ; 0,03\]
y confianza en el rango de \[0,05 ; 0,1\]

\section{Análisis}
En esta sección se analizan las mejores reglas generadas en la sección
de procesamiento de datos. Se muestran reglas que representan gustos
de películas, se intenta establecer reglas relacionadas con la
posición demográfica, como también analizar similitudes entre los
los rankings de peliculas en el año 2000 y 2001.

\subsection{Reglas generales}
La primera impresión al ver las reglas generadas por Apriori, es que todas 
referencian a ratings positivos, siendo muy pocas las relacionadas con 
calificaciones mediocres o bajas. Es por esto que se analiza la frecuencia
de estas 3 calificaciones en la tabla \ref{rating_freq}, donde puede
claramente notarse que existe una tendencia a calificar las 
películas de forma elevada.

\begin{table}[ht!]
\caption{Frecuencia de ratings}
\label{rating_freq}
\centering
\begin{tabular}{l c}
Calificación & Frecuencia \\
low &  16\% \\
medium & 26\% \\
high & 58\%
\end{tabular}
\end{table}

\subsubsection{Directores mejor rankeados}
La tabla \ref{table_best_directors} muesta las reglas relativas a los 
directores de películas mejor rankeados.
\begin{table}[ht!]
\caption{Reglas relativas a directores}
\label{table_best_directors}
\centering
\begin{tabular}{l l l l }
regla & sop. & conf. & lift \\
\{Steven Spielberg\} $=$$>$ \{high\} & 2,36\% & 90\% & 1,15 \\
\{Alfred Hitchcock\} $=$$>$ \{high\} & 1,23\% & 95\% & 1,22 \\
\{James Cameron\} $=$$>$ \{high\} & 1,20\% & 89\% & 1,14 \\
\{Rob Reiner\} $=$$>$ \{high\} & 1,17\% & 92\% & 1,18
\end{tabular}
\end{table}

El soporte del director mejor rankeado, Steven Spielberg, es de un 2,36\%. Aunque a nivel
porcentual este soporte parece bajo, a nivel nominal significa que de 888.500 ratings
totales, en 21.000 aparecen Steven Spielberg y de esas 18.900 la calificaron como un
rating \{high\}. Los soportes de la tabla \ref{table_best_directors} 
son nominalmente significativo. 

\subsubsection{Actores mejor rankeados}
La tabla \ref{table_best_cast}   muestra las reglas relativas a los actores 
mejor ranqueados.
\begin{table}[ht!]
\caption{Reglas relativas a actores individuales}
\label{table_best_cast}
\centering
\begin{tabular}{l l l l }
regla & sop. & conf. & lift \\
\{Harrison Ford\} $=$$>$ \{high\} & 2,85\% & 91\% & 1,17 \\
\{Tom Hanks\} $=$$>$ \{high\} & 1,96\% & 89\% & 1,15 \\
\{Robert De Niro\} $=$$>$ \{high\} & 1,53\% & 87\% & 1,11 \\
\{Sean Connery\} $=$$>$ \{high\} & 1,35\% & 85\% & 1,09 \\
\{Arnold Schwarzenegger\} $=$$>$ \{high\} & 1,17\% & 77\% & 0,99
\end{tabular}
\end{table}


\subsubsection{dupla de actores mejor ranqueados}
La tabla \ref{table_best_tuple}   muestra las reglas relativas a la 
dupla de  actores mejor ranqueados.
\begin{table}[ht!]
\caption{Reglas relativas a dupla de actores}
\label{table_best_tuple}
\centering
\begin{tabular}{l l l l }
regla & sop. & conf. & lift \\
\{Carrie Fisher, Harrison Ford\} $=$$>$ \{high\} & 1,10\% & 95\% & 1,22 \\
\{Harrison Ford, Mark Hamill\} $=$$>$ \{high\} & 1,10\% & 95\% & 1,22 \\
\{Billy Dee Williams, Carrie Fisher\} $=$$>$ \{high\} & 0,70\% & 94\% & 1,20 \\
\end{tabular}
\end{table}

Esta tabla confirma aún mas que Harrison Ford es uno de los actores preferidos.

\subsubsection{Generos}
La tabla \ref{table_genre} muestra los generos mejores ranqueados.
\begin{table}[ht!]
\caption{Reglas relativas a actores individuales}
\label{table_genre}
\centering
\begin{tabular}{l l l l }
regla & sop. & conf. & lift \\
\{Drama\} $=$$>$ \{high\} & 23\% & 64\% & 1,11 \\
\{Comedy\} $=$$>$ \{high\} & 20\% & 54\% & 0,95 \\
\{Action\} $=$$>$ \{high\} & 14\% & 53\% & 0,93 \\
\{Romance\} $=$$>$ \{high\} & 9\% & 57\% & 0,99 \\
\end{tabular}
\end{table}

La base de datos contiene 18 categorias. En caso de que haya independencia en 
los gustos de géneros, el soporte de cada categoría debería tender a 5,55\%. 
Estos niveles de soporte, superiores al 10\% y algunos cercanos al 20\%,
implican que hay una preferencia sobre los géneros de drama, comedia,
accion y romance.

\subsubsection{A los hombres les gusta las peliculas de guerra}
La tabla \ref{table_male_ware} muestra que a los hombres
les gusta las películas de guerra, con o sin drama. Sin embargo,
hay una sensible preferencia por las peliculas de guerra sin
drama.

\begin{table}[ht!]
\caption{Reglas relacionadas con peliculas de guerra}
\label{table_male_ware}
\centering
\begin{tabular}{l l l l }
regla & sop. & conf. & lift \\
\{male, war\} $=$$>$ \{high\} & 3,75\% & 69\% & 1,20 \\
\{Drama, male, war\} $=$$>$ \{high\} & 2,51\% & 75\% & 1,29 \\
\end{tabular}
\end{table}

\subsubsection{El declive de la décade de los ' 90}
Se han generado pocas reglas con calificaciones negativas. Una de las
que llama la atención es que muchas de ellas son relacionadas con películas
de la década de los ' 90, tal como muestra en la tabla \ref{table_year_90}.
\begin{table}[ht!]
\caption{El declive de los ' 90}
\label{table_year_90}
\centering
\begin{tabular}{l l l l }
regla & sop. & conf. & lift \\
\{25$-$34, low, male\} $=$$>$ \{1990\} & 3,44\% & 64\% & 1,20 \\
\{Comedy, low, male\} $=$$>$ \{1990\} & 3,07\% & 65\% & 1,21 \\
\{Drama, low\} $=$$>$ \{1990\} & 2,77\% & 67\% & 1,21 \\
\{Action, low, male\} $=$$>$ \{1990\} & 2,74\% & 66\% & 1,25 
\end{tabular}
\end{table}

\subsubsection{Las personas de 45 a 55, son quienes mejores califican}
La tabla \ref{table_age} muestra las reglas de las edades de los que
mejor califican. Probablemente no sea casualidad que ambas categorías 
son consegutivas.
\begin{table}[ht!]
\caption{Mejores calificaciones según la edad}
\label{table_age}
\centering
\begin{tabular}{l l l l }
regla & sop. & conf. & lift \\
\{45$-$49\} $=$$>$ \{high\} & 4,50\% & 62\% & 1,08 \\
\{50$-$55\} $=$$>$ \{high\} & 4,49\% & 59\% & 1,03 \\
\end{tabular}
\end{table}

\subsubsection{Generos en común}
La tabla \ref{table_common_genre} muestra generos relacionados 
entre si. 
\begin{table}[ht!]
\caption{Generos en común}
\label{table_common_genre}
\centering
\begin{tabular}{l l l l }
regla & sop. & conf. & lift \\
\{Action,high,Sci-Fi\} $=$$>$ \{Adventure\} & 2,53\% & 51\% & 3,80 \\
\{high,Sci-Fi,Thriller\} $=$$>$ \{Action\} & 1,93\% & 76\% & 2,94 
\end{tabular}
\end{table}





\subsection{Reglas demográficas}
En esta sección se analiza las relaciones y preferencias según las variables 
demográficas estado y ciudad. 

\subsubsection{Estados y peliculas}
La tabla \ref{table_state_population} hace una comparación entre que porcentaje
de votos hicieron los estados que más calificaron y el porcentaje de habitantes
que tiene el estado respecto del total del país.

\begin{table}[ht!]
\caption{Votos y porcentaje poblacional}
\label{table_state_population}
\centering
\begin{tabular}{l l l}
estado & \% votos & \%población  \\
California & 18,07 & 11,78 \\
New York & 7,08 & 6,13 \\
Minnesota & 6,45 & 1,70 \\
Illinois & 5,50 & 4,07 \\
Texas & 5,23 & 8,36
\end{tabular}
\end{table}

Estos resultados muestran, principalmente, que tanto el estado de California
como el de Minesotta, son outlayers severos. Mientras que la población de
Minesota es del 1,70\% de la población de USA, el 6,45\% de los votos pertenecen
a este estado. El ratio es cercano a 4 veces más su población, lo que demuestra
que el estado de Minnesota tiene una preferencia de entretenerse viendo películas.
Algo similar sucede con el estado de Californa, donde su población es de casi el
12\% de los Estados Unidos, pero el 18\% de los votos totales pertenecen a este
estado. California también es un outlayer, pero este se comprende mas ya que la
mayor industria cinematográfica de Estados Unidos es Hollywood, que se encuentra
en el estado de California.

Otro de los puntos destacados que muestra la table \ref{table_state_population} es
que el estado de Texas no es propenso a entretenerse mirando películas.

Otra regla interesante es que el estado que mejor califica películas es California.

\begin{table}[ht!]
\centering
\begin{tabular}{l l l l }
regla & sop. & conf. & lift \\
\{high\} $=$$>$ \{CA\} & 10,44\% & 57\% & 1,03 \\
\{MN\} $=$$>$ \{HIGH\} & 3,81\% & 59\% & 1,03 
\end{tabular}
\end{table}

Este sesgo posiblemente también se deba a que Hollywood pertenece a California,
y mucha gente que vive de esta industria o se ve influenciada por su cercanía
física, 
esta calificando de forma positiva las películas. Es consistente que la regla
\{MN\} $=$$>$ \{HIGH\} tiene un soporte del 3,81\%, respecto del 10,44\% de 
California, ya que Minesotta no se encuentra influenciada por Hollywood.

\subsubsection{ciudades y peliculas}
La tabla \ref{citi_votes} muestra las ciudades que más votos realizaron y
la tabla \ref{citi_high} muestra las reglas de ciudades con mas calificaciones
positivas. Ambas tablas son consistentes entre si, y también son consistentes
con el análisis de los estados. Lo que llama la atención es que la ciudad de Los
Angeles, la 2da ciudad mas  pobladas de los Estados Unidos, 
ciudad que también pertenece al estado de California, 
\textbf{no} esta incluida entre las que mas votan\footnote{Los Angeles tiene
un 1\% de los votos totales}, como tampoco se pudieron generar reglas que 
incluyan a esta ciudad y a algúna calificación. Esta inexistencia
de reglas para esta ciudad particular, implicaría una "pseudo-regla" que
refleja que a la ciudad de Los Angeles \textbf{no} le gusta ver películas.
 
\begin{table}[ht!]
\caption{Votos por ciudades}
\label{citi_votes}
\centering
\begin{tabular}{l l }
ciudad & \% votos \% \\
Minneapolis & 2,52\% \\
New York & 2,37\% \\
San Francisco & 2,12\% \\
Saint Paul & 2,09\%  \\
\end{tabular}
\end{table}

\begin{table}[ht!]
\caption{Votos positivos por ciudad}
\label{citi_high}
\centering
\begin{tabular}{l l l l }
regla & sop. & conf. & lift \\
\{Minneapolis\} $=$$>$ \{high\} & 1,48\% & 58\% & 1,02 \\
\{New York\} $=$$>$ \{high\} & 1,38\% & 58\% & 1,01 \\
\{San Francisco\} $=$$>$ \{high\} & 1,22\% & 57\% & 0,99 \\
\end{tabular}
\end{table}


\subsubsection{Estados y géneros}
En esta sección se analiza la relaciones existentes entre estados y géneros
de películas. La tabla \ref{table_genre_state} resume la reglas mas relevantes al respecto.

\begin{table}[ht!]
\caption{Reglas mas relevantes sobre géneros y estados}
\label{table_genre_state}
\centering
\begin{tabular}{l l l l l }
nro & regla & sop. & conf. & lift \\
I & \{\} $=$$>$ \{Drama\} & 36\% & 36\% & 1 \\
II & \{\} $=$$>$ \{Comedy\} & 35\% & 36\% & 1 \\
III & \{\} $=$$>$ \{Action\} & 26\% & 26\% & 1 \\
IV & \{\} $=$$>$ \{Thriller\} & 19\% & 19\% & 1 \\
V & \{CA\} $=$$>$ \{Drama\} & 6,69\% & 37\% & 1,02 \\
VI & \{CA\} $=$$>$ \{Comedy\} & 6,32\% & 34\% & 0,97 \\
VII & \{CA\} $=$$>$ \{Action\} & 4,70\% & 26\% & 1,00 \\
VIII & \{CA\} $=$$>$ \{Thriller\} & 3,48\% & 19\% & 1,14 \\
IX & \{CA,high\} $=$$>$ \{Drama\} & 4,31\% & 41\% & 1,42 \\
X & \{CA,high\} $=$$>$ \{Comedy\} & 3,53\% & 33\% & 0,94 \\
\end{tabular}
\end{table}

La 1ra a 4ta regla muestra que en todo Estados Unidos, 
la gente mira príncipalmente pelícuas de drama y comedia, 
en 3er medida de acción y en 4ta  Thriller. 
La 4ta a 8va regla se muestran las asociaciones mas fuertes que del 
lado izquierdo pertenece a un estado, y el derecho a un género
de película. Las 4 reglas tienen un soporte relativamente significativo y
pertenecen todas al mismo estado: California, estado con mayores votaciones.
Lo más interesante, es que los géneros de películas más vistos en California,
coinciden con los de todo Estados Unidos, tanto en orden, como en distancia
relativa entre cada uno de ellos.
La 9ena y 10ma regla muestra las asociaciones con mayores soporte y confianza, cuyo
lado izquierdo tiene un estado y una calificación elevada, y el lado derecho
un género. Los resultados, no por casualidad, coinciden con las películas
mas vistas en California, como en todo Estados Unidos.

El análisis de la tabla \ref{table_genre_state} nos permite concluir que
California es un estado dominante en cuestión de género de peliculas. 

Sería interesante analizar si esta misma posición dominante se repite en el pasado.
Si llegase a ser así, pero con diferentes géneros, se podría pensar que la industria
cinematográfica de Holliwood sigue los gustos de California, y un cambio en los 
gustos de ese estado, impactaría en el tipo de peliculas que Estados Unidos produciría.

La tabla \ref{table_genre_state_high} muestra las reglas de asociación mas fuerte que
destacan los géneros mejores calificados para cada estado relevante.

\begin{table}[ht!]
\caption{Generos mejores calificados por estado}
\label{table_genre_state_high}
\centering
\begin{tabular}{l l l l }
regla & sop. & conf. & lift \\
\{CA,Drama\} $=$$>$ \{high\} & 4,31\% & 64\% & 1,11 \\
\{CA,Comedy\} $=$$>$ \{high\} & 3,53\% & 56\% & 0,97 \\
\{Drama,NY\} $=$$>$ \{high\} & 1,79\% & 63\% &  1,13 \\
\{Comedy,NY\} $=$$>$ \{high\} & 1,36\% & 54\% &  0,98 \\
\{Drama,MN\} $=$$>$ \{high\} & 1,50\% & 65\% &  1,07 \\
\{Comedy,MN\} $=$$>$ \{high\} & 1,37\% & 56\% &  0,88 \\
\{Drama,IL\} $=$$>$ \{high\} & 1,21\% & 61\% & 1,13 \\
\{Comedy,IL\} $=$$>$ \{high\} & 1,01\% & 50\% & 0,98 \\
\{Drama,TX\} $=$$>$ \{high\} & 1,22\% & 65\% & 1,15 \\
\{Comedy,TX\} $=$$>$ \{high\} & 1,11\% & 56\% & 0,98 \\
\{Drama,MA\} $=$$>$ \{high\} & 1,12\% & 66\% & 1,15 \\
\{Comedy,MA\} $=$$>$ \{high\} & 1,00\% & 58\% & 1,02 \\
\end{tabular}
\end{table}

Esta tabla refuerza aún más la posición dominante que tiene California sobre la industria
cinematográfica. En todos los estados mas destacados, los géneros de 
las peliculas mas vistas coinciden con los de California. La diferencia es el nivel de
soporte de estas reglas respecto a porcentaje poblacional e inclusive al porcentaje de
votos en esta encuenta. California tiene porcentajes relativos mucho mas fuertes, avalando
la hipótesis de que Hollywood produce películas del agrado de California, y el resto de
los estados mira las películas realizadas por su mayor productor.


\subsubsection{Ciudades y géneros}
Al intentar encontrar la existencia de relaciones entre ciudades y géneros, se
han encontrado las reglas de la tabla \ref{citi_genre}.

\begin{table}[ht!]
\caption{Generos mas vistos por ciudad}
\label{citi_genre}
\centering
\begin{tabular}{l l l l }
regla & sop. & conf. & lift \\
\{New York\} $=$$>$ \{Drama\} & 0,99\% & 42\% & 1,63 \\
\{Minneapolis\} $=$$>$ \{Drama\} & 0,93\% & 37\% & 1,02 \\
\{Seattle\} $=$$>$ \{High\} & 1,12\% & 63\% & 1,10 \\
\end{tabular}
\end{table}

Estas reglas no aportan información relevante, solo refuerzan el desarrollo
previo. Posiblemente la imposibilidad de encontrar una fuerte relación entre
ciudades y géneros se debe a 2 motivos:
\begin{itemize}
\item El nivel de granularidad de agrupamiento por ciudad es elevado.
\item Los géneros de películas mas vistas estan muy concentrados. Los 3 géneros
mas importantes se llevan más del 70\% de la torta, haciendo insignificante la
producción de otros géneros. Si los géneros producidos fuesen estuviesen
distribuidos más uniformemente, 
posiblemente pueda esperarse que a la ciudad de Utah tenga preferencias por las
películas religiosas, los de New York por musicales y Las Vegas por películas 
para adultos.
\end{itemize}





\subsection{Gustos del año 2000 vs. 2001}
TO BE DONE

\section{Conclusiones}
TO BE DONE

\section{Posibles mejoras}
Normalizar las calificaciones POR USUARIO
Hablar de crear una DB
Hablar de probar con Neo4J

\section*{REFERENCES}

\end{document}
